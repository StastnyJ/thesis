\section{Memory Consumption}\label{section:memoryConsumption}

Especially for the robots with low-performant controllers, for example, older Raspberry PI, RAM is a very limited resource, so the memory needed for storing the local views needs to be as small as possible. Therefore another metric evaluated in the algorithms is the consumption of the memory. The measurement results are shown in the table \ref{tab:memory}.\par

\begin{table}[htpb]
    \caption{Average memory consumption of the algorithms in the different environments}\label{tab:memory}
    \centering
    \begin{tabular}{l | l  l| l l| l l}
        \toprule
        \textbf{}          & \multicolumn{2}{l|}{\textbf{1st stage only}} & \multicolumn{2}{l|}{\textbf{both stages}} & \multicolumn{2}{l}{\textbf{OpenRatSLAM}}                                                             \\
        {}                 & $\oslash$ LV size                            & $\oslash$ LVs stored                      & $\oslash$ LV size                        & $\oslash$ LVs st. & $\oslash$ LV size & $\oslash$ LVs st. \\
        \hline
        \textbf{Warehouse} & 78 B                                         & 194                                       & 1102 B                                   & 191               & 600 B             & 312               \\
        \textbf{House}     & 128 B                                        & 219                                       & 1152 B                                   & 216               & 600 B             & 445               \\
        \textbf{Hospital}  & 130 B                                        & 151                                       & 1154 B                                   & 148               & 600 B             & 219               \\
        \bottomrule
    \end{tabular}
\end{table}

The OpenRatSLAM uses a compressed 60x10 pixels big grayscale image as a local view template, so the memory needed for storing a single local view is always constant. However, the memory required for storing a single LV in the approaches suggested in this work differs for each local view and depends on the number of clusters detected by the DBScan algorithm. The difference between the size of the LV templates used in the approach with only the first stage and with both stages differs by a feature vector of 256 floating point numbers. Therefore, as follows from the table, the memory consumption of the first stage-only approach is significantly smaller than while also using the second stage.\par
The memory needed for storing the local views using the 2-stage approach is, on average, almost twice as large as the memory required for storing the lv template in the OpenRatSLAM approach. However, the number of stored templates in the OpenRatSLAM is almost twice larger than the number of stored templates in the 2-stage approach, so the total memory consumption remains similar. More interesting is the algorithm with only the first stage, in which a single local view consumes about six times less memory than the OpenRatSLAM approach. Furthermore, this approach stored, on average, significantly fewer local views than the OpenRatSLAM algorithm, so the total memory consumption is up to 12 times lower than in the OpenRatSLAM.
