\chapter{\abstractname}

Simultaneous localization and mapping is a crucial problem for most autonomous mobile robots. RatSLAM is a biologically inspired SLAM system that simulates the rodent's brain using the Pose Cell network, odometry, and visual input. This system demonstrates long-term stability and accurate results in indoor and outdoor environments. One of the essential parts of RatSLAM is a scene recognition algorithm. Whereas RatSLAM uses a visual scene recognition approach, this work suggests a technique to combine data from 3D LiDAR and camera to achieve better scene recognition results and hence improve the whole SLAM system. Compared to the camera, the LiDAR sensor is more accurate and robust to changing light conditions. This thesis proposes several scene recognition approaches using combined data from the camera and 3D LiDAR sensor. These approaches are combined with RatSLAM to build an improved biologically inspired SLAM system. All proposed methods are tested on performance using several evaluation metrics and compared with visual scene recognition and OpenRatSLAM. As the results show, the suggested methods are capable of accurate and robust scene recognition, and final SLAM systems are able to construct suitable experience maps. Furthermore, all proposed approaches outperformed the OpenRatSLAM in most of the evaluated metrics and worked smoothly also in situations where OpenRatSLAM failed.


\makeatletter
\ifthenelse{\pdf@strcmp{\languagename}{english}=0}
{\renewcommand{\abstractname}{Kurzfassung}}
{\renewcommand{\abstractname}{Abstract}}
\makeatother

\chapter{\abstractname}

\begin{otherlanguage}{ngerman}

    Simultaneous localization and mapping ist ein entscheidendes Problem für die meisten autonomen mobilen Roboter. RatSLAM ist ein biologisch inspiriertes SLAM-System, das das Gehirn des Nagetiers mithilfe des Pose Cell-Netzwerks, Odometrie und visueller Eingabe simuliert. Dieses System zeigt langfristige Stabilität und genaue Ergebnisse in Innen- und Außenumgebungen. Einer der wesentlichen Bestandteile von RatSLAM ist ein Szenenerkennungsalgorithmus. Während RatSLAM einen visuellen Szenenerkennungsansatz verwendet, schlägt diese Arbeit eine Technik vor, um Daten von 3D-LiDAR und Kamera zu kombinieren, um bessere Ergebnisse bei der Szenenerkennung zu erzielen und somit das gesamte SLAM-System zu verbessern. Im Vergleich zur Kamera ist der LiDAR-Sensor genauer und robuster gegenüber wechselnden Lichtverhältnissen. Diese Arbeit schlägt mehrere Ansätze zur Szenenerkennung vor, die kombinierte Daten von der Kamera und dem 3D-LiDAR-Sensor verwenden. Diese Ansätze werden mit RatSLAM kombiniert, um ein verbessertes biologisch inspiriertes SLAM-System aufzubauen. Alle vorgeschlagenen Methoden werden anhand mehrerer Bewertungsmetriken auf ihre Leistung getestet und mit visueller Szenenerkennung und OpenRatSLAM verglichen. Wie die Ergebnisse zeigen, sind die vorgeschlagenen Verfahren zu einer genauen und robusten Szenenerkennung in der Lage, und endgültige SLAM-Systeme sind in der Lage, geeignete Erfahrungskarten zu erstellen. Darüber hinaus übertrafen alle vorgeschlagenen Ansätze OpenRatSLAM in den meisten der bewerteten Metriken und funktionierten reibungslos auch in Situationen, in denen OpenRatSLAM versagt hat.

\end{otherlanguage}


\makeatletter
\ifthenelse{\pdf@strcmp{\languagename}{english}=0}
{\renewcommand{\abstractname}{Abstract}}
{\renewcommand{\abstractname}{Kurzfassung}}
\makeatother